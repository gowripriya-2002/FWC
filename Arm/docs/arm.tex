\def\mytitle{Arm  Assignment}
\def\myauthor{A.Gowri Priya}
\def\contact{gowripriyaappayyagari@gmail.com}
\def\mymodule{Future Wireless Communication (FWC)}
\documentclass[10pt, a4paper]{article}
\usepackage[a4paper,outer=1.5cm,inner=1.5cm,top=1.75cm,bottom=1.5cm]{geometry}
\twocolumn
\usepackage{setspace}
\usepackage{graphicx}
\graphicspath{{./images/}}
\usepackage[colorlinks,linkcolor={black},citecolor={blue!80!black},urlcolor={blue!80!black}]{hyperref}
\usepackage[parfill]{parskip}
\usepackage{lmodern}
\usepackage{tikz}
	\usepackage{physics}
%\documentclass[tikz, border=2mm]{standalone}
\usepackage{karnaugh-map}
\usepackage{tabularx}
\usetikzlibrary{calc}
\usepackage{amsmath}
\usepackage{amssymb}
\renewcommand*\familydefault{\sfdefault}
\usepackage{watermark}
\usepackage{lipsum}
\usepackage{xcolor}
\usepackage{listings}
\usepackage{float}
\usepackage{titlesec}
\providecommand{\mtx}[1]{\mathbf{#1}}
\titlespacing{\subsection}{1pt}{\parskip}{3pt}
\titlespacing{\subsubsection}{0pt}{\parskip}{-\parskip}
\titlespacing{\paragraph}{0pt}{\parskip}{\parskip}

\newcommand{\myvec}[1]{\ensuremath{\begin{pmatrix}#1\end{pmatrix}}}
\let\vec\mathbf
\lstset{
frame=single, 
breaklines=true,
columns=fullflexible
}
\title{\mytitle}
\author{\myauthor\hspace{1em}\\\contact\\FWC22013\hspace{6.5em}IITH\hspace{0.5em}\mymodule\hspace{6em}ASSIGNMENT}
\date{}
\begin{document}
	\maketitle
\section{Problem}
Reduce the following Boolean Expression to its simplest form using K-Map :
E(U,V,Z,W)=   (2 , 3 , 6 , 8 , 9 , 10 , 11 , 12 , 13 )


\section{Components}

\begin{center}
    \setlength{\arrayrulewidth}{0.1mm}
	\setlength{\tabcolsep}{12pt}
	\renewcommand{\arraystretch}{1.5}
    \begin{tabular}{|c|c|c|}
    \hline 
    \textbf{S.No} & \textbf{Component} & \textbf{Number}\\ \hline
	1. & Vaman Board & 1 \\
	2. & Bread Board & 1 \\
	3. & Jumer Wires(F-M) & 10 \\
	4. & LED & 1 \\
	5. & Resistor(150 ohm) & 1 \\ \hline   
   \end{tabular}
\end{center}

\section{K-Map}
    \paragraph{}
 From the given data the minterms are\\ 2,3,6,8,9,10,11,12,13.
     
\begin{karnaugh-map}[4][4][1][][]
    \maxterms{0,1,4,5,7,14,15}
    \minterms{2,3,6,8,9,10,11,12,13}

    % note: posistion for start of \draw is (0, Y) where Y is
    % the Y size(number of cells high) in this case Y=2
    \draw[color=black, ultra thin] (0, 4) --
    node [pos=0.7, above right, anchor=south west] {$XW$} % Y label
    node [pos=0.7, below left, anchor=north east] {$ZY$} % X label
    ++(135:1);
        
\end{karnaugh-map}

The minimized expression is
    E=(UZ'+V'Z+U'ZW')

   
    \begin{karnaugh-map}[4][4][1][][]
         \maxterms{0,1,4,5,7,14,15}
         \minterms{2,3,6,8,9,10,11,12,13}
        \implicantedge{3}{2}{11}{10}
        \implicantedge{8}{9}{12}{13}
        \implicant{2}{6}
    \draw[color=black, ultra thin] (0, 4) --
    node [pos=0.7, above right, anchor=south west] {$XW$} % Y label
    node [pos=0.7, below left, anchor=north east] {$ZY$} % X label
    ++(135:1);
\end{karnaugh-map}

\section{Truth Table}

\begin{center}
   \begin{tabularx}{0.5\textwidth} {
  | >{\centering\arraybackslash}X
  | >{\centering\arraybackslash}X
  | >{\centering\arraybackslash}X
  | >{\centering\arraybackslash}X
  | >{\centering\arraybackslash}X | }
\hline
 U& V & Z & W & E \\
\hline
0 & 0 & 0 & 0 & 0\\ 
\hline
0 & 0 & 0 & 1& 0 \\
\hline
0 & 0 & 1 & 0 & 1\\
\hline
0 & 0 & 1 & 1 & 1 \\
\hline
0 & 1 & 0 & 0 & 0 \\ 
\hline
0 & 1 & 0 & 1 & 0 \\
\hline
0 & 1 & 1 & 0 & 1 \\
\hline
0 & 1 & 1& 1 & 0 \\
\hline
1 & 0 & 0 & 0 & 1\\ 
\hline
1 & 0 & 0 & 1& 1\\
\hline
1 & 0 & 1 & 0 & 1\\
\hline
1 & 0 & 1 & 1 & 1 \\
\hline
1 & 1 & 0 & 0 & 1 \\ 
\hline
1 & 1 & 0 & 1 & 1 \\
\hline
1 & 1 & 1 & 0 & 0 \\
\hline
1 & 1 & 1& 1 & 0 \\
\hline
\end{tabularx}
Truth Table
\end{center}

\section{Procedure}
\raggedright 1.After executing the following code using make, a binary file is generated with .bin extension in the output directory.  \\  2.Now from the termux, using scp protocol, send the generated bin file to the laptop. \\ \vspace{2mm}
\raggedright 3.There we are supposed to flash the .bin file into the ARM through the terminal.\\ \vspace{2mm}
\raggedright 4.After flashing, reset the Vaman board.\\ \vspace{2mm}
\raggedright 5.Make connections between the LED and ARM board using jumper wires. \\ \vspace{2mm}
\raggedright 6.Now check the output with reference to the truth table present above.
\section{Execution}
*Verify the above truth table by using the minimized expression in the following code.
\framebox{
\url{https://github.com/gowripriya-2002/FWC/blob/main/Arm/Codes/src/main.c}}
\bibliographystyle{ieeetr}
\section{Conclusion}
Hence the  given boolean expression is minimized and  verified  it's functionality by using ARM.
\end{document}
